\documentclass{dcbl/challenge}

\setdoctitle{Der Titel meines Dokuments}
\setdocauthor{Stephan Bökelmann}
\setdocemail{sboekelmann@ep1.rub.de}
\setdocinstitute{AG Physik der Hadronen und Kerne}


\begin{document}

This worksheet is designed to guide you through setting up GitHub Actions workflows for your C projects. Automating the compilation process helps in identifying errors early and streamlines your development process. You will learn to create workflows for both simple compilation and projects using Makefiles.
A GitHub workflow is a fancy way of saying: Perform a declared task on a virtual machine. If you configure your worklow correctly you can automate a ton of stuff. The biggest Usecase is for sure to automate testing code. 
You can find an example project here: \url{https://github.com/PhilippTheSurfer/C-CI_template}
Please feel free to fork the repository and play around.

\section*{Aufgaben}
\begin{aufgabe}
\textbf{Setting Up a Simple Compilation Workflow}

Create a new GitHub repository and add a simple C program. For example, you can use a "Hello, World!" program. 

\textbf{Instructions:}
\begin{enumerate}
    \item In your repository, create a directory named \texttt{.github/workflows}.
    \item Inside this directory, create a file named \texttt{ci.yml}.
    \item Add the following content to the \texttt{ci.yml} file to set up your workflow:
\begin{verbatim}
name: C Compile Workflow

on:
  push:
    branch:
      - main

jobs:
    build:
    runs-on: ubuntu-latest
    
    steps:
    - uses: actions/checkout@v2
    - name: Install GCC
        run: sudo apt-get update && sudo apt-get install gcc
    - name: Compile C program
        run: gcc hello_world.c -o hello_world
    - name: Run C program
        run: ./hello_world
\end{verbatim}
    \item Adjust the compile command if your file has a different name than \texttt{hello_world.c}.
    \item Commit and push your changes. GitHub Actions will automatically trigger the workflow on every push to the branch master. If you want to change when the workflow is triggerd just declare a different branchname in the on push section.
\end{enumerate}
\end{aufgabe}

\begin{aufgabe}[Für Fortgeschrittene]
\textbf{Setting Up a Workflow with Make}

Assuming your project uses a Makefile for build automation, follow these steps to create a workflow that compiles your project using Make.
\textit{Hint: In my example repository I just reused the makefile provided in the previous paper.}

\textbf{Instructions:}
\begin{enumerate}
    \item Ensure your repository contains your C source files and a Makefile.
    \item In the \texttt{.github/workflows} directory, create a file named \texttt{make.yml}.
    \item Populate \texttt{make.yml} with the following workflow:
\begin{verbatim}
name: Make Build Workflow

on:
  push:
    branch:
      - main

jobs:
    build:
    runs-on: ubuntu-latest
    
    steps:
    - uses: actions/checkout@v2
    - name: Setup Build Environment
        run: sudo apt-get update && sudo apt-get install -y build-essential

    - name: Build Project
        run: make build
        working-directory: path/to/your/makefile
    
    - name: Install Project
        run: sudo make install
        working-directory: path/to/your/makefile
  
    - name: Clean Up
        run: make clean
        working-directory: path/to/your/makefile
        

\end{verbatim}
    \item Replace \texttt{path/to/your/makefile} with the relative path to your Makefile if it's not in the root directory.
    \item Commit and push the \texttt{make.yml} file to your repository. GitHub Actions will compile your project using Make on each push.
\end{enumerate}
\end{aufgabe}

\section*{Notes}
\begin{enumerate}
    \item Remember to check the "Actions" tab in your GitHub repository to see the status and logs of your workflow runs.
    \item Workflow files are YAML files and are sensitive to indentation. Ensure you use spaces consistently and follow the YAML syntax correctly.
    \item For more complex projects, you might need to customize your workflows further, such as by adding additional dependencies or setting environment variables.
\end{enumerate}

\end{document}
